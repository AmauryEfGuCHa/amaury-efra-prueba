\documentclass[12pt,a4paper]{article}
\usepackage[utf8]{inputenc}
\usepackage[spanish]{babel}
\usepackage{amsmath}
\usepackage{amsfonts}
\usepackage{amssymb}
\title{Tarea: Cinematica directa e inversa de manipuladores paralelos}
\author{Amaury Efrain Gutierrez Chavez}
\begin{document}
\maketitle
Para los manipuladores paralelos la cinemática inversa resulta ser más simple, en comparación con la cinemática directa.\\
\\
En robots paralelos la cinemática inversa consiste en establecer el valor de las juntas activas y pasivas en función de las coordenadas del extremo del robot, las juntas activas son las juntas actuadas y las juntas pasivas son las que quedan en función de las juntas activas. Establecer la cinemática inversa es esencial para el control de la posición de los robots paralelos. La cinemática inversa de los manipuladores paralelos en general es idéntica a la de los manipuladores serie.\\
\\
A diferencia de la cinemática inversa, en la cinemática directa existe el problema es determinar la posición del efector final en función de las juntas activas. Hay diferentes tipos de soluciones. En el estudio de la cinemática directa Merlet, Tsai, Ángeles, Raghavan, recurrieron a la solución de un polinomio mostraron que el problema de la cinemática directa es reducir la solución a polinomio.


\end{document}